\chapter{Desarrollo de la solución}\label{CAP6}

\section{Diagrama de flujo}
        \newpage

\section{Software ROS}
    explicar que se ara en esta seccion funciones orden y de donde se sacan sistema SI y orden 
        \newpage

    \subsection{Metodología A}

 \subsubsection{Cinemática directa}
    \begin{algorithm}[H]
        \caption{Cinemática Directa Metodología A} 
        \SetKwInput{KwInput}{Input}                % Set the Input
        \SetKwInput{KwOutput}{Output}              % set the Output
        \SetKwInput{KwObjetivo}{Objetivo}                % Set the Input
        \SetKwInput{Kwfile}{Nombre archivo}                % Set the Input
        
        \DontPrintSemicolon
          \KwObjetivo {Hallar la posición del efector final del robot delta dada una configuración articular}
          \Kwfile{$delta.kinematics.t1m.adams.py$}
          \KwInput{$\theta_1,\theta_2,\theta_3$}
          \KwOutput{$E_0(x_0,y_0,z_0)$}
          
          % Set Function Names{\theta }_1,{\theta }_2,{\theta }_3
          \SetKwFunction{FSub}{forward}
           \SetKwProg{Fn}{}{:}{}
        
          \tcc{FUNCION PRINCIPAL}
            \Fn{\FSub{${\theta }_1$,${\theta }_2$,${\theta }_3$}}{
                \tcc{Calcular Centros de esferas (Ecuaciones tabla [\ref{tab:cap4_tabla_3}])}
                $(x_1,y_1,z_1)$=${J^'}_1\left(0,\left[-\frac{f-e}{2\sqrt{3}}-r_f{\mathrm{cos} \left({\theta }_1\right)\ }\right],-r_f{\mathrm{sin}\mathrm{n} \left({\theta }_1\right)\ }\right)$\;
                $(x_2,y_2,z_2)$=${J^'}_2(\left[\frac{f-e}{2\sqrt{3}}+r_f{\mathrm{cos} \left({\theta }_2\right)\ }\right]\mathrm{cos}\mathrm{}(30{}^\circ ),\left[\frac{f-e}{2\sqrt{3}}+r_f{\mathrm{cos} \left({\theta }_2\right)\ }\right]\mathrm{sin}\mathrm{}(30{}^\circ ),-r_f{\mathrm{sin} \left({\theta }_2\right)\ })$\;
                $(x_3,y_3,z_3)$=${J^'}_3(-\left[\frac{f-e}{2\sqrt{3}}+r_f{\mathrm{cos} \left({\theta }_3\right)\ }\right]\mathrm{cos}\mathrm{}(30{}^\circ ),\left[\frac{f-e}{2\sqrt{3}}+r_f{\mathrm{cos} \left({\theta }_3\right)\ }\right]\mathrm{sin}\mathrm{}(30{}^\circ ),-r_f{\mathrm{sin} \left({\theta }_3\right)\ })$\;
                \tcc{Calcular intersección de las 3 esferaS (Ecuaciones de la \ref{eq:cap4_eq_3} a la \ref{eq:cap4_eq_13})}
                $d= \left(y_2- y_1\right)x_3-\left(y_3-y_1\right)x_2- \left(y_2-y_3\right)x_1$\;
                $w_1={x_1}^2+{y_1}^2 +{z_1}^2$\;
                $w_2={x_2}^2+{y_2}^2 +{z_2}^2$\;
                $w_3={x_3}^2+{y_3}^2 +{z_3}^2$\;
                $a_1=\frac{\left(z_2-z_1\right)\left(y_3-y_1\right)-\left(z_3-z_1\right)\left(y_2-y_1\right)}{d}$\;
                $b_1=\left(\frac{1}{2*(-d)}\right)*\left(\left(w_2 - w_1\right)\left(y_3-y_1\right)-\left(w_3 - w_1\right)\left(y_2-y_1\right)\right)$\;
                $a_2=\frac{-1}{d}*\left[\left(z_2-z_1\right)x_3-(z_3-z_1)x_2+(z_3-z_2)x_1\right]$\;
                $b_2=\frac{1}{2d}*[\left(w_2-w_1\right)x_3-\left(w_3-w_1\right)x_2+\left(w_3-w_2)\right)x_1$\;
                $A = {a_1}^2+{a_2}^2+1$\;
                $B = 2(a_1(b_1-x_1)+a_2(b_2-y_1)-z_1)$\;
                $C=  ({(b_1-x_1)}^2+{(b_2-y_1)}^2+{z_1}^2- {r_e}^2)$\;
                $E_0(x_0,y_0,z_0) =  \left(a_1z+\ b_1,a_2z+\ b_2,\frac{-B-\ \sqrt{\left(B^2\right)-\left(4AC\right)}}{2A}\right)$\;
                           \KwRet  $(x_0,y_0,z_0)$\;
                }
    \end{algorithm}
    
    \newpage
 
 \subsubsection{Cinemática inversa}
     \begin{algorithm}[H]
            \caption{Cinemática Inversa Metodología A} 
            \SetKwInput{KwInput}{Input}                % Set the Input
            \SetKwInput{KwOutput}{Output}              % set the Output
            \SetKwInput{KwObjetivo}{Objetivo}                % Set the Input
            \SetKwInput{Kwfile}{Nombre archivo}                % Set the Input
            
            \DontPrintSemicolon
              \KwObjetivo {Hallar la posición de los actuadores en el espacio articular dada la posición del centroide del efector final}
              \Kwfile{$delta.kinematics.t1m.adams.py$}
              \KwInput{$E_0(x_0,y_0,z_0)$}
              \KwOutput{${\theta }_1$,${\theta }_2$,${\theta }_3$}
              
                % Set Function Names
              \SetKwFunction{FSum}{inverse}
              \SetKwFunction{FSub}{$angle.yz$}
            
              \tcc{FUNCION PRINCIPAL}
              \SetKwProg{Fn}{}{:}{}
              \Fn{\FSum{$x_0,y_0,z_0$}}{
              \tcc{Rotación del sistema de referencia Local en $0^{\circ}$}
                ${x}_{0 ^{\circ}}=x_0$\;
                ${y}_{0 ^{\circ}}=y_0$\;
                ${z}_{0 ^{\circ}}=z_0$\;
              \tcc{Rotación del sistema de referencia Local en $120^{\circ}$}
                ${x}_{120 ^{\circ}}=x_0*cos120+y_0*sin120$\;
                ${y}_{120 ^{\circ}}=y_0*cos120-x_0*sin120$\;
                ${z}_{120 ^{\circ}}=z_0$\;
              \tcc{Rotación del sistema de referencia Local en $240^{\circ}$}
                ${x}_{240 ^{\circ}}=x_0*cos120-y_0*sin120$\;
                ${y}_{240 ^{\circ}}=y_0*cos120+x_0*sin120$\;
                ${z}_{240 ^{\circ}}=z_0$\;
                
              \tcc{Calcular angulos de los actuadores}
              \tcc{Cambiar orden de ángulos al sistema Referencia Global}
                 ${\theta }_2=angle.yz({x}_{0 ^{\circ}} , {y}_{0 ^{\circ}} , {z}_{0 ^{\circ}},0 ^{\circ})$\;    
                 ${\theta }_3=angle.yz({x}_{120 ^{\circ}},{y}_{120 ^{\circ}},{z}_{120 ^{\circ}},120 ^{\circ})$\;
                 ${\theta }_1=angle.yz({x}_{240 ^{\circ}},{y}_{240 ^{\circ}},{z}_{240 ^{\circ}},240 ^{\circ})$\;
            
               \KwRet $[{\theta }_1,{\theta }_2,{\theta }_3]$\;
              }
    \end{algorithm}
    
\newpage

    \begin{algorithm}
          \ContinuedFloat
          \caption{Cinemática Inversa Metodología A (Continuacion...)}
          \tcc{SUBFUNCIONES}
          \SetKwProg{Fn}{}{:}{}
          \Fn{\FSub{${x}_{{ \phi }_i} , {y}_{{ \phi }_i} , {z}_{{ \phi }_i}$, { \phi }_i}}{
               \tcc{Actuador $F_{i}$ en el plano $Y'Z'$ (sistema de referencia Local rotado en ${ \phi }_i$)}
                $y_{1}= y_{F_{i}} = -\frac{f}{2\sqrt[]{3}}$\;
                $z_{1}=z_{F_{i}}=0$\;        
               \tcc{ Proyección de la Junta Esférica $E_{i}$ en el plano $Y'Z'$(sistema de referencia Local rotado en ${ \phi }_i$)}    
                $y_{2}= y_{E'_{i}} = {y}_{{ \phi }_i}-\frac{e}{2\sqrt[]{3}}$\;
                $z_{2}=z_{E'_{i}}={z}_{{ \phi }_i} $\;
                \tcc{Junta Esférica $J_{i}$ de la cadena cinemática i en posición ${ \phi }_i$}
                $a=\frac{{x}_{{ \phi }_i}^{2}+y_{2}^{2}+{z}_{{ \phi }_i}^{2}+r_{f}^{2}- r_{e}^{2}~-y_{1}^{2}}{2{z}_{{ \phi }_i}}$\;
                $b=\frac{ \left( y_{1}-y_{2} \right) }{z_{2}}$\;
                $x_{J_{i}}=0$\;
                $y_{J_{i}}=$ $\frac{(y_{1}-ab )-\sqrt{ (a+by_{1}) ^{2}+ (b^{2}+1)r_{f}^{2}}}{b^{2}+1}$\;
                $z_{J_{i}}=a+b*y_{J_{i}}$\;
                \tcc{ Calculo del Angulo ${\theta }_i$ de la cadena cinemática i en posición ${ \phi }_i$}
                ${\theta }_i=\arctan( ~\frac{z_{J_{i}}}{y_{F_{i}}-y_{J_{i}}})$\;
                \KwRet${\theta }_i$\;
          }
    \end{algorithm}

 
\newpage
  
        
    \subsubsection{Jacobiano}
            
        \begin{algorithm}[H]
            \caption{Jacobiano Metodología A} 
            \SetKwInput{KwInput}{Input}                % Set the Input
            \SetKwInput{KwOutput}{Output}              % set the Output
            \SetKwInput{KwObjetivo}{Objetivo}                % Set the Input
            \SetKwInput{Kwfile}{Nombre archivo}                % Set the Input
            
              \DontPrintSemicolon
              \KwObjetivo {Hallar la matriz jacobiana}
              \Kwfile{$jacobian.tm1.adams.py$}
              \KwInput{$P_x,P_y,P_z,{\theta }_{11},{\theta }_{12},{\theta }_{13}$}
              \KwOutput{ $J_{\theta }, J_{x },\theta _{31},\theta _{32},\theta _{33},\theta _{21},\theta _{22},\theta _{23}, J$}
               
                % Set Function Names,
              \SetKwFunction{FSum}{jacobian.total}
              \SetKwFunction{FSub}{calculo.theta3i}
            
              \tcc{FUNCION PRINCIPAL}
              \SetKwProg{Fn}{}{:}{}
              \Fn{\FSum{$P_x$,$P_y$,$P_z$,${\theta }_{11}$,${\theta }_{12}$,${\theta }_{13}$}}{
                  \tcc{Cambiar ángulos y punto P al sistema Referencia Local}
                  \tcc{Posición actuadores $\phi$ respecto a ejes YX S.ref Local}
                   $({\phi}_{0 ^{\circ}},{\phi}_{120 ^{\circ}},{\phi}_{240 ^{\circ}})=(0,120,240)$\;
                  \tcc{Calcular angulo $\theta _{3i}$ }
                  $\theta _{31}=calculo.theta3i(P_x,P_y, {\phi}_{0 ^{\circ}})$\;
                  $\theta _{32}=calculo.theta3i(P_x,P_y, {\phi}_{120 ^{\circ}})$\;
                  $\theta _{33}=calculo.theta3i(P_x,P_y, {\phi}_{240 ^{\circ}})$\;
                  \tcc{Calcular angulo $\theta _{2i}$ }
                  $\theta _{21}=calculo.theta2i(P_x,P_y,P_z,{\theta }_{31}, {\phi}_{0 ^{\circ}})$\;
                  $\theta _{22}=calculo.theta2i(P_x,P_y,P_z,{\theta }_{32}, {\phi}_{120 ^{\circ}})$\;
                  $\theta _{23}=calculo.theta2i(P_x,P_y,P_z,{\theta }_{33}, {\phi}_{240 ^{\circ}})$\;
                  \tcc{Calcular Matriz $J_{x}$ }
                  $ J_{1x}=calculo.jix($\theta _{11},\theta _{21},\theta _{31},${ \phi }_{0 ^{\circ}})$\;
                  $ J_{2x}=calculo.jix($\theta _{12},\theta _{22},\theta _{32},${ \phi }_{120 ^{\circ}})$\;
                  $ J_{3x}=calculo.jix($\theta _{13},\theta _{23},\theta _{33},${ \phi }_{240 ^{\circ}})$\;
                  $ J_{1y}=calculo.jiy($\theta _{11},\theta _{21},\theta _{31},${ \phi }_{0 ^{\circ}})$\;
                  $ J_{2y}=calculo.jiy($\theta _{12},\theta _{22},\theta _{32},${ \phi }_{120 ^{\circ}})$\;
                  $ J_{3y}=calculo.jiy($\theta _{13},\theta _{23},\theta _{33},${ \phi }_{240 ^{\circ}})$\; 
                   $ J_{1z}=calculo.jiz($\theta _{11},\theta _{21},\theta _{31},${ \phi }_{0 ^{\circ}})$\;
                  $ J_{2z}=calculo.jiz($\theta _{12},\theta _{22},\theta _{32},${ \phi }_{120 ^{\circ}})$\;
                  $ J_{3z}=calculo.jiz($\theta _{13},\theta _{23},\theta _{33},${ \phi }_{240 ^{\circ}})$\; 
                  $ J_{x}=[J_{1x},J_{1y},J_{1z};J_{2x},J_{2y},J_{2z};J_{3x},J_{3y},J_{3z}]$\;
                  \tcc{Calcular Matriz $J_{ \theta }$ }
                    $J_{1 \theta }=a~\sin  \theta _{21}~sin~ \theta _{31}$\;
                    $J_{2 \theta }=a~\sin  \theta _{22}~sin~ \theta _{32}$\;  
                    $J_{3 \theta }=a~\sin  \theta _{23}~sin~ \theta _{33}$\;  
                    $J_{ \theta }=[J_{1 \theta },0,0;0,J_{2 \theta },0;0,0,J_{3 \theta }]$\;  
                \tcc{Calcular Matriz $J$ }
                    $J=J_{x}^{-1}J_{ \theta } $\;
                \tcc{Cambiar al sistema Referencia Global}
        
                  \KwRet $[J_{\theta }, J_{x},\theta _{31},\theta _{32},\theta _{33},\theta _{21},\theta _{22},\theta _{23}, J]$
                  }
        \end{algorithm}
        
\newpage

\begin{algorithm}
          \ContinuedFloat
          \caption{Jacobiano Metodología A (Continuacion...)}
          \tcc{SUBFUNCIONES}
          
          \tcc{Calcular angulo ${\theta }_{3i}$}  
          \SetKwProg{Fn}{}{:}{}
          \Fn{\FSub{$P_x$,$P_y$ ,$\phi_i$}}{
            $c_{yi}= (-P_x)*(sin({ \phi }_i))+(P_y)*(cos({ \phi }_i))$\;    
            $\theta _{3i}= \cos ^{-1}\frac{c_{yi}}{b}$\;    
                \KwRet${\theta }_{3i}$\;   } 
        
          \tcc{Calcular angulo ${\theta }_{2i}$}
          \SetKwProg{Fn}{}{:}{}
          \SetKwFunction{FSub}{calculo.theta2i}
          \Fn{\FSub{$P_x$,$P_y$,$P_z$,${\theta }_{3i}$,$\phi_i$}}{
                      	$c_{xi}=(P_x)*(cos({ \phi }_i))+(P_y)*(sin({ \phi }_i))+ h-r$\; 
            	$c_{yi}= (-P_x)*(sin({ \phi }_i))+(P_y)*(cos({ \phi }_i))$\; 
        	    $c_{zi}=P_z$\; 
        
                $\theta _{2i}=\cos ^{-1} \left( \frac{c_{xi}^{2}+c_{yi}^{2}+c_{zi}^{2}-a^{2}-b^{2}~}{2ab sin~ \theta _{3i}} \right)$\;        
                \KwRet${\theta }_{2i}$\;  }
                
                
           \tcc{Calcular componentes Matriz $J_x$}
           \SetKwFunction{FSub}{calculo.jix}
           \SetKwProg{Fn}{}{:}{}
           \Fn{\FSub{$\theta _{1i}$,$\theta _{2i}$,$\theta _{3i}$, $\phi_i$}}{
            $ J_{ix}=\cos  \left(  \theta _{1i}+ \theta _{2i} \right) sin~ \theta _{3i}\cos  \phi _{i}-cos  \theta _{3i}\sin  \phi _{i}~ $\;    
            \KwRet$J_{ix}$\;  }
                
                
          \SetKwFunction{FSub}{calculo.jiy}
           \SetKwProg{Fn}{}{:}{}
          \Fn{\FSub{$\theta _{1i}$,$\theta _{2i}$,$\theta _{3i}$,$\phi_i$}}{
            $J_{iy}=\cos  \left(  \theta _{1i}+ \theta _{2i} \right) sin~ \theta _{3i}\sin  \phi _{i}+ cos  \theta _{3i}\cos  \phi _{i}~$\;    
            \KwRet$J_{iy}$\;  }   
                    
            
           \SetKwFunction{FSub}{calculo.jiz}
           \SetKwProg{Fn}{}{:}{}
          \Fn{\FSub{$\theta _{1i}$,$\theta _{2i}$,$\theta _{3i}$,$\phi_i$}}{
            $ J_{iz}=sin \left(  \theta _{1i}+ \theta _{2i} \right) sin~ \theta _{3i}~ $\;    
            \KwRet$J_{iz}$\;} 
            
          \tcc{Calcular componentes Matriz $J_{\theta }$}
          \SetKwFunction{FSub}{calculo.jthetai}
          \Fn{\FSub{$\theta_{2i}$,$\theta _{3i}$,$\phi_i$}}{
                $J_{i \theta }=a~\sin  \theta _{2i}~sin~ \theta _{3i}$\; 
                \KwRet$J_{i \theta }$\;   }
\end{algorithm}

        \newpage
        \subsubsection{Dinámica Inversa}
        
        
        
        
        
        
        
        
        
        
        
        
        
        
        
        
        
        
        
        
        \newpage

        
    \subsection{Metodología B}
        \subsubsection{Cinemática directa}
        
\begin{algorithm}[H]
    \caption{Cinemática Directa Metodología B} 
    \SetKwInput{KwInput}{Input}                % Set the Input
    \SetKwInput{KwOutput}{Output}              % set the Output
    \SetKwInput{KwObjetivo}{Objetivo}                % Set the Input
    \SetKwInput{Kwfile}{Nombre archivo}                % Set the Input
    
    \DontPrintSemicolon
      \KwObjetivo {Hallar la posición del efector final del robot delta dada una configuración articular}
      \Kwfile{$delta.kinematics.Paderborn.tm1.adams$}
      \KwInput{$\theta_1,\theta_2,\theta_3$}
      \KwOutput{$P_0(P_{0x},P_{0y},P_{0z})$}
      
      % Set Function Names{\theta }_1,{\theta }_2,{\theta }_3
      \SetKwFunction{FSub}{forward.Paderborn}
       \SetKwProg{Fn}{}{:}{}
    
      \tcc{FUNCION PRINCIPAL}
        \Fn{\FSub{${\theta }_1$,${\theta }_2$,${\theta }_3$}}{
            \tcc{Calcular Centros de esferas (Ecuaciones (\ref{eq:cap4_MB_3}),(\ref{eq:cap4_MB_4}),(\ref{eq:cap4_MB_5}))}
            \tcc{Ordenar angulos segun sistema de referencia Local}
            
            $(x_1,y_1,z_1)$=$\overrightarrow{J'_{2}} \left [  0,-\frac{(f-e)}{2\sqrt{3}}-L_{A}cos(\theta_2),-L_{A}sin(\theta_2)\right]$\;
            
            $(x_2,y_2,z_2)$=$\overrightarrow{J'_{3}} \left [\left( \frac{(f-e)}{2\sqrt{3}}+{L}_{A}cos(\theta_3)\right) cos(30^\circ), \left(\frac{(f-e)}{2\sqrt{3}} + {L}_{A}cos(\theta_3)\right) sin(30^\circ), -L_{A}sin(\theta_3)\right]$\;
            
            $(x_3,y_3,z_3)$=$\overrightarrow{J'_{1}} \left [\left( \frac{(f-e)}{2\sqrt{3}}+{L}_{A}cos(\theta_1)\right) cos(30^\circ), \left(\frac{(f-e)}{2\sqrt{3}} + {L}_{A}cos(\theta_1)\right) sin(30^\circ), -L_{A}sin(\theta_1)\right]$\;
            
            \tcc{Calcular intersección de las 3 esferaS (Ecuaciones de la \ref{eq:cap4_eq_3} a la \ref{eq:cap4_eq_13})}
            $d= \left(y_2- y_1\right)x_3-\left(y_3-y_1\right)x_2- \left(y_2-y_3\right)x_1$\;
            $w_1={x_1}^2+{y_1}^2 +{z_1}^2$\;
            $w_2={x_2}^2+{y_2}^2 +{z_2}^2$\;
            $w_3={x_3}^2+{y_3}^2 +{z_3}^2$\;
            $a_1=\frac{\left(z_2-z_1\right)\left(y_3-y_1\right)-\left(z_3-z_1\right)\left(y_2-y_1\right)}{d}$\;
            $b_1=\left(\frac{1}{2*(-d)}\right)*\left(\left(w_2 - w_1\right)\left(y_3-y_1\right)-\left(w_3 - w_1\right)\left(y_2-y_1\right)\right)$\;
            $a_2=\frac{-1}{d}*\left[\left(z_2-z_1\right)x_3-(z_3-z_1)x_2+(z_3-z_2)x_1\right]$\;
            $b_2=\frac{1}{2d}*[\left(w_2-w_1\right)x_3-\left(w_3-w_1\right)x_2+\left(w_3-w_2)\right)x_1$\;
            $A = {a_1}^2+{a_2}^2+1$\;
            $B = 2(a_1(b_1-x_1)+a_2(b_2-y_1)-z_1)$\;
            $C=  ({(b_1-x_1)}^2+{(b_2-y_1)}^2+{z_1}^2- {r_e}^2)$\;
            $P_0(P_{0x},P_{0y},P_{0z}) =  \left(a_1z+\ b_1,a_2z+\ b_2,\frac{-B-\ \sqrt{\left(B^2\right)-\left(4AC\right)}}{2A}\right)$\;
                       \KwRet  $(P_{0x},P_{0y},P_{0z})$\;
            }
\end{algorithm}

\newpage        
 
        
     
        
        \newpage

\subsubsection{Cinemática inversa}
        
    \begin{algorithm}[H]
            \caption{Cinemática Inversa Metodología B} 
            \SetKwInput{KwInput}{Input}                % Set the Input
            \SetKwInput{KwOutput}{Output}              % set the Output
            \SetKwInput{KwObjetivo}{Objetivo}                % Set the Input
            \SetKwInput{Kwfile}{Nombre archivo}                % Set the Input
            
            \DontPrintSemicolon
              \KwObjetivo {Hallar la posición de los actuadores en el espacio articular dada la posición del centroide del efector final}
              \Kwfile{$delta.kinematics.Paderborn.tm1.adams$}
              \KwInput{$P_0(P_{0x},P_{0y},P_{0z})$}
              \KwOutput{${\theta }_1$,${\theta }_2$,${\theta }_3$}
              
                % Set Function Names
              \SetKwFunction{FSum}{inverse.Paderborn}
              \SetKwFunction{FSub}{$angle.yz.Paderborn$}
            
              \tcc{FUNCION PRINCIPAL}
              \SetKwProg{Fn}{}{:}{}
              \Fn{\FSum{$P_{0x},P_{0y},P_{0z}$}}{
              \tcc{Rotación del sistema de referencia Local en $0^{\circ}$}
                ${x}_{0 ^{\circ}}=P_{0x}$\;
                ${y}_{0 ^{\circ}}=P_{0y}$\;
                ${z}_{0 ^{\circ}}=P_{0z}$\;
              \tcc{Rotación del sistema de referencia Local en $120^{\circ}$}
                ${x}_{120 ^{\circ}}=P_{0x}*cos120+P_{0y}*sin120$\;
                ${y}_{120 ^{\circ}}=P_{0y}*cos120-P_{0x}*sin120$\;
                ${z}_{120 ^{\circ}}=P_{0z}$\;
              \tcc{Rotación del sistema de referencia Local en $240^{\circ}$}
                ${x}_{240 ^{\circ}}=P_{0x}*cos120-P_{0y}*sin120$\;
                ${y}_{240 ^{\circ}}=P_{0y}*cos120+P_{0x}*sin120$\;
                ${z}_{240 ^{\circ}}=P_{0z}$\;
                
              \tcc{Calcular angulos de los actuadores}
              \tcc{Cambiar orden de ángulos al sistema Referencia Global}
                 ${\theta }_2=angle.yz.Paderborn({x}_{0 ^{\circ}} , {y}_{0 ^{\circ}} , {z}_{0 ^{\circ}},0 ^{\circ})$\;    
                 ${\theta }_3=angle.yz.Paderborn({x}_{120 ^{\circ}},{y}_{120 ^{\circ}},{z}_{120 ^{\circ}},120 ^{\circ})$\;
                 ${\theta }_1=angle.yz.Paderborn({x}_{240 ^{\circ}},{y}_{240 ^{\circ}},{z}_{240 ^{\circ}},240 ^{\circ})$\;
            
               \KwRet $[{\theta }_1,{\theta }_2,{\theta }_3]$\;
              }
    \end{algorithm}
    
    \clearpage   
        
\begin{algorithm}
      \ContinuedFloat
      \caption{Cinemática Inversa Metodología B (Continuacion...)}
      \tcc{SUBFUNCIONES}
      \SetKwProg{Fn}{}{:}{}
      \Fn{\FSub{${x}_{{ \varphi }_i} , {y}_{{ \varphi }_i} , {z}_{{ \varphi }_i}$, { \varphi }_i}}{
           \tcc{Actuador $F_{i}$ en el plano $Y'Z'$ (sistema de referencia Local rotado en ${ \varphi }_i$)}
            $y_{1}= y_{F_{i}} = -\frac{f}{2\sqrt[]{3}}$\;
            $z_{1}=z_{F_{i}}=0$\;        
           \tcc{ Proyección de la Junta Esférica $P_{0}$ en el plano $Y'Z'$(sistema de referencia Local rotado en ${ \varphi }_i$)}    
            $y_{2}= y_{P'_{0}} = {y}_{{ \varphi }_i}-\frac{e}{2\sqrt[]{3}}$\;
            $z_{2}=z_{P'_{0}}={z}_{{ \varphi }_i} $\;
            \tcc{Junta Esférica $J_{i}$ de la cadena cinemática i en posición ${ \varphi }_i$}
            $a=\frac{{x}_{{ \varphi }_i}^{2}+y_{2}^{2}+{z}_{{ \varphi }_i}^{2}+R_{A}^{2}- R_{B}^{2}~-y_{1}^{2}}{2{z}_{{ \varphi }_i}}$\;
            $b=\frac{ \left( y_{1}-y_{2} \right) }{z_{2}}$\;
            $x_{J_{i}}=0$\;
            $y_{J_{i}}=$ $\frac{(y_{1}-ab )-\sqrt{ (a+by_{1}) ^{2}+ (b^{2}+1)r_{f}^{2}}}{b^{2}+1}$\;
            $z_{J_{i}}=a+b*y_{J_{i}}$\;
            \tcc{ Calculo del Angulo ${\theta }_i$ de la cadena cinemática i en posición ${ \varphi }_i$}
            ${\theta }_i=\arctan( ~\frac{z_{J_{i}}}{y_{F_{i}}-y_{J_{i}}})$\;
            \KwRet${\theta }_i$\;
      }
\end{algorithm}
    
        \newpage

 \subsubsection{Jacobiano}
     
         \begin{algorithm}[H]
            \caption{Jacobiano Metodología B} 
            \SetKwInput{KwInput}{Input}                % Set the Input
            \SetKwInput{KwOutput}{Output}              % set the Output
            \SetKwInput{KwObjetivo}{Objetivo}                % Set the Input
            \SetKwInput{Kwfile}{Nombre archivo}                % Set the Input
            
            \DontPrintSemicolon
              \KwObjetivo {Hallar la matriz jacobiana}
              \Kwfile{$jacobian.Paderborn.tm1.v2.adams.py$}
              \KwInput{$P_{0x},P_{0y},P_{0z},{\theta }_{1},{\theta }_{2},{\theta }_{3}$}
              \KwOutput{ $-J_{1 }, J_{2}, J$}
               
                % Set Function Names,
              \SetKwFunction{FSum}{jacobian.calculo.Paderborn}

              \tcc{FUNCION PRINCIPAL}
              \SetKwProg{Fn}{}{:}{}
              \Fn{\FSum{$P_x$,$P_y$,$P_z$,${\theta }_{11}$,${\theta }_{12}$,${\theta }_{13}$}}{
              \tcc{Cambiar ángulos y punto P al sistema Referencia Local}
              \tcc{Posición actuadores $\varphi$ respecto a ejes $YX$ S.ref Local}
              $({\phi}_{0 ^{\circ}},${\phi}_{120 ^{\circ}},${\phi}_{240 ^{\circ}})=(0,120,240)$\;

              \tcc{Calcular  $\overrightarrow{s_{i}}$ Transpose  }
              $\overrightarrow{s_{1}}^T={si(P_{0x},P_{0y},P_{0z},{\theta }_{1},\varphi_{0 ^{\circ}})}^T$\;
              $\overrightarrow{s_{2}}^T={si(P_{0x},P_{0y},P_{0z},{\theta }_{2},\varphi_{120 ^{\circ}})}^T$\;
              $\overrightarrow{s_{3}}^T={si(P_{0x},P_{0y},P_{0z},{\theta }_{3},\varphi_{240 ^{\circ}})}^T$\;
              \tcc{Calcular $\overrightarrow{b_{i}}$  }
              $\overrightarrow{b_{1}}={bi({\theta }_{1},\varphi_{0 ^{\circ}})}$\;
              $\overrightarrow{b_{2}}={bi({\theta }_{2},\varphi_{120 ^{\circ}})}$\;
              $\overrightarrow{b_{3}}={bi({\theta }_{3},\varphi_{240 ^{\circ}})}$\;
              
              \tcc{Calcular componentes $J_{1}$ }
              $J_{1}$=\left.    
              {\begin{bmatrix}
                 $\overrightarrow{s_{1}}^T(0)$ & $\overrightarrow{s_{1}}^T(1)$ & $\overrightarrow{s_{1}}^T(2)$\\
                $\overrightarrow{s_{2}}^T(0)$ & $\overrightarrow{s_{2}}^T(1)$ & $\overrightarrow{s_{2}}^T(2)$\\
                $\overrightarrow{s_{3}}^T(0)$ & $\overrightarrow{s_{3}}^T(1)$ & $\overrightarrow{s_{3}}^T(2)$ 
            \end{bmatrix}}^{-1}
            \right.

              \tcc{Calcular $J_{2}$ }
              $J_{2}$=\left.    
              \begin{bmatrix}
                 $\overrightarrow{s_{1}}^T \cdot \overrightarrow{b_{1}} $ & $0$ & $0$\\
                $0$ & $\overrightarrow{s_{2}}^T \cdot \overrightarrow{b_{2}}$ & $0$\\
                $0$ & $0$ & $\overrightarrow{s_{3}}^T \cdot \overrightarrow{b_{3}}$
            \end{bmatrix}
            \right.

              \tcc{Calcular $J$ ecuacion (\ref{eq:cap4_MB_14}) }
                $J=-J_{1}J_{2}$\;
                
               
            \tcc{Cambiar al sistema Referencia Global}
    
              \KwRet $[-J_{1},J_{2},J]$   }
        \end{algorithm}
 

\begin{algorithm}
          \ContinuedFloat
          \caption{Jacobiano Metodología B (Continuacion...)}
          \tcc{SUBFUNCIONES}
          
          \tcc{Calcular $\overrightarrow{s_{i}}$ }
          \SetKwProg{Fn}{}{:}{}
          \SetKwFunction{FSub}{si}
          \Fn{\FSub{$P_{0x}$,$P_{0y}$,$P_{0z}$,${\theta }_{i}$,$\varphi_i$}}{
                \tcc{Vector efector final }
                $\overrightarrow{P_{0}}=
                \left.          
                \begin{bmatrix}
                P_{0x} \\
                P_{0y} \\
                P_{0z}
            \end{bmatrix}  \right.$\;
                \tcc{Matriz de rotacion respecto a ángulo 
               $\varphi_i$ }
            $R_{i}^{R}=matri.rot(\varphi_i)$\;
            \tcc{Vector posición actuador i respecto a sistema referencia local }
            $R={R}_{A}-{R}_{B}$\;
            $ \overrightarrow{F_{i}}=R_{i}^{R} 
                \left.          
                \begin{bmatrix}
               R \\
                0 \\
                0
            \end{bmatrix}  \right.$\;
            \tcc{Vector brazo i respecto a sistema referencia local }
             $\overrightarrow{\xi_{F_{i}J_{i}}}=R_{i}^{R}
              \left.          
                \begin{bmatrix}
               L_{A} cos(\theta_i) \\
                0 \\
                -L_{A} sin(\theta_i)
            \end{bmatrix}  \right.$\;
            \tcc{calculo $\overrightarrow{s_{i}}$ ecuacion (\ref{eq:cap4_MB_12})}
            $ \overrightarrow{s_{i}} & = \overrightarrow{P_{0}}- \left(\overrightarrow{F_{i}}+\overrightarrow{\xi_{F_{i}J_{i}}} \right)$\;
            \KwRet$\overrightarrow{s_{i}}$\;  }
                
          \tcc{Calcular $\overrightarrow{b_{i}}$ }
          \SetKwProg{Fn}{}{:}{}
          \SetKwFunction{FSub}{bi}
          \Fn{\FSub{${\theta }_{i}$,$\varphi_i$}}{
                \tcc{Matriz de rotacion respecto a ángulo 
               $\varphi_i$ }
            $R_{i}^{R}=matri.rot(\varphi_i)$\;
            \tcc{calculo $\overrightarrow{b_{i}}$ ecuacion (\ref{eq:cap4_MB_15}) }
            $ \overrightarrow{b_{i}} =  R_{i}^{R}      \begin{bmatrix}
                L_{A} sin(\theta_i) \\
                0\\
                L_{A} cos(\theta_i) 
            \end{bmatrix}$\;
            \KwRet$\overrightarrow{b_{i}}$\;  }
        
          \tcc{Calcular Matriz de rotacion $R_{i}^{R}$ ecuacion (\ref{eq:cap4_MB_13})}
          \SetKwProg{Fn}{}{:}{}
          \SetKwFunction{FSub}{matri.rot}
          \Fn{\FSub{$\varphi_i$}}{
            $ R_{i}^{R} =
        \begin{bmatrix}
                cos(\varphi_i)&-sin(\varphi_i)&0 \\
                sin(\varphi_i)&cos(\varphi_i)&0 \\
                0&0&1
            \end{bmatrix}$\;
            \KwRet$R_{i}^{R}$\;  }
\end{algorithm}

        \newpage

\subsubsection{Dinámica Inversa}
        \newpage

        
    \subsection{Espacio de Trabajo}
     Encontrar el espacio de trabajo de un robot delta significa calcular todas las posiciones posibles y viables del efector final. El siguiente diagrama de flujo representa una de las muchas soluciones que existen, donde las dimensiones del robot y los parámetros de las restricciones son valores preestablecidos. Es importante recalcar que la nomenclatura en esta sección es la misma que se implementa para determinar el jacobiano del robot delta de la sección \ref{ma_jac}.  
    
             \begin{figure}[htb]
                \centering
                \includegraphics[width=1\linewidth]{Main/Chapter6/Images6/cap6_ws_2.png}
                \caption{Fotografía de un paraguas}
                \label{f:Cap6_ws_2}
            \end{figure}  

    CAMBIARDIAGRAMA La idea principal de la solución propuesta es calcular un grupo significativo de configuraciones $(P_x,P_y,P_z)$ posibles del efector final en base a sus dimensiones y evaluar si cumplen con restricciones impuestas. Se obtienen las configuraciones cartesianas del efector $(P_x,P_y,P_z)$  utilizando la cinemática directa y variando los angulos $\theta_{11}$,$\theta_{12}$,$\theta_{13}$ en los rangos $[\theta_{11min}-\theta_{11max}]$, $[\theta_{12min}-\theta_{12max}]$, $[\theta_{13min}-\theta_{13max}]$ respectivamente. Todos estos rangos se discretizan con un paso impuesto \(  \Delta  \theta _{1i} \). Si el punto  $(P_x,P_y,P_z)$ cumple con todas las restricciones, el punto pertenece al espacio de trabajo. Al contrario, si el punto no cumple con una o mas restricciones, este no pertenece al espacio de trabajo.

        \newpage

    Las restricciones son las mismas que las expuestas en la sección \ref{restriccionesWS}. El orden en que se verifican las restricciones se muestran en el siguiente diagrama de flujo:

             \begin{figure}[htb]
                \centering
                \includegraphics[width=1\linewidth]{Main/Chapter6/Images6/cap6_ws_3.png}
                \caption{Fotografía de un paraguas}
                \label{f:Cap6_ws_3}
            \end{figure}  


    Para calcular el espacio de trabajo se crea un nodo en ROS. La tarea especifica de este nodo es calcular, guardar y graficar los puntos $(P_x,P_y,P_z)$ del efector final que cumplen con las restricciones impuestas. 
    La representación gráfica del nodo se presenta en la ilustración \ref{f:Cap6_ws_1}: 
    
                \begin{figure}[htb]
                \centering
                \includegraphics[width=0.9\linewidth]{Main/Chapter6/Images6/cap6_ws_1.png}
                \caption{Fotografía de un paraguas}
                \label{f:Cap6_ws_1}
            \end{figure}  
    
    
    En este nodo, las dimensiones del robot y las restricciones son establecidas antes de calcular el espacio de trabajo. La única entrada es el paso de la discretizacion de los rangos \(  \Delta  \theta _{1i} \). La cinematica directa y el jacobiano se calculan a partir del método A, específicamente los desarrollados en las secciones \ref{ma_cd} y \ref{ma_jac} respectivamente. Se emplea este método por la razón de que el jacobiano $J$ se puede bipartir en dos matrices $(J_{x}$  y  $J_{ \theta })$, donde cada una de ellas representa una singularidad especifica . Ademas, el metodo A tiene la ventaja que tiene las ecuaciones para determinar los ángulos interiores $\theta_{2i}$ y $\theta_{3i}$.
    
    \newpage
    
    El valor de los parámetros de las restricciones impuestas en este trabajo de grado se presentan y se explican en la tabla \ref{t:cap6_ws_1}: 
    
            \begingroup
            \renewcommand{\arraystretch}{1.5}
            \begin{table}[H]
            \centering
            \begin{tabular}{c m{7.5cm} c c}
               \hline
               \textbf{Restriccion}  & \textbf{Explicacion} & \textbf{Min}& \textbf{Max}\\
               \hline           \hline            
             $\theta_{1i}$ & Colisión entre los brazos y la base fija & $-90^{\circ}$ & $90^{\circ}$\\
            \hline
             $\Delta\theta _{1i}$ & Pasos de discretizacion de rangos $\theta_{1i}$ muy bajos implica costo computacional alto& $1^{\circ}$ & $90^{\circ}$ \\
            \hline
             $\theta _{2i}$ & Angulo debe ser menor a $180^{\circ}$ ya que en esa situación el brazo y antebrazo podrían ser colineales y producir singularidades. Ademas esta restriccion asegura que el calculo de la cinemática de posición sea la correcto ya que descarta una de las soluciones & $5^{\circ}$ & $175^{\circ}$ \\
            \hline
             $\theta _{3i}$ & Angulo respecto a la inclinación máxima de las rotulas por catalogo, generalmente son de $13^{\circ}$ & $75^{\circ}$ & $165^{\circ}$ \\
            \hline
             $J_{x}$ & Depende de la precisión o factor de seguridad subjetivo& $5*10^{-1}$ & $\inf$ \\
            \hline
             $J_{\theta}$ & Depende de la precisión o factor de seguridad subjetivo& $1*10^{-4}$ & $\inf$ \\
            \hline
             $X$ & Limite X de espacio de trabajo impuesto por el fabricante & $-100[mm]$ & $+100[mm]$ \\
            \hline            
             $Y$ & Limite Y de espacio de trabajo impuesto por el fabricante & $-100[mm]$ & $+100[mm]$ \\
            \hline   
             $Z$ & Limite Z de espacio de trabajo impuesto por el fabricante & $-100[mm]$ & $+100[mm]$ \\
            \hline   
            \end{tabular}
            \caption{Referencias del dibujo}
            \label{t:cap6_ws_1}
        \end{table}
        \endgroup     
        
         \begin{figure}[htb]
                \centering
                \includegraphics[width=0.1\linewidth]{Main/Chapter6/Images6/cap6_ws_4.png}
                \caption{Fotografía de un paraguas}
                \label{f:Cap6_ws_4}
            \end{figure}    
    
        \newpage

    El siguiente algoritmo presenta el pseudocódigo de la función dentro del nodo que determinar el espacio de trabajo:
    
\begin{algorithm}[H]
    \caption{Espacio de Trabajo} 
    \SetKwInput{KwInput}{Input}                % Set the Input
    \SetKwInput{KwOutput}{Output}              % set the Output
    \SetKwInput{KwObjetivo}{Objetivo}                % Set the Input
    \SetKwInput{Kwfile}{Nombre archivo}                % Set the Input
    
    \DontPrintSemicolon
      \KwObjetivo {Encontrar el espacio de trabajo del robot delta a partir de sus dimensiones y restricciones impuestas.}
      \Kwfile{$workspace.py$}
      \KwInput{ $\Delta\theta _{1i}$}
      \KwOutput{$[\mathcal{W}_{(J_{x},J_{\theta})},\mathcal{W}_{J_{x}},\mathcal{W}_{J_{\theta}},\mathcal{W}_{(J_{x},J_{\theta},Limites)}]$}
      
      % Set Function Names{\theta }_1,{\theta }_2,{\theta }_3
      \SetKwFunction{FSub}{espaciotrabajo}
       \SetKwProg{Fn}{}{:}{}
    
      \tcc{FUNCION PRINCIPAL}
        \Fn{\FSub{$\Delta\theta _{1i}$}}{
            \tcc{Barrido de angulos $\theta _{11}$ , $\theta _{12}$ y $\theta _{13}$ con incrementos de $\Delta\theta _{1i}$}
            \For{\theta _{11} \in [$\theta_{11min}$ - $\theta_{11max}]$;$\Delta\theta _{1i}$}    
                { 
                \For{\theta _{12} \in [$\theta_{12min}$ - $\theta_{12max}]$;$\Delta\theta _{1i}$}  
                    { 
                    \For{\theta _{13} \in [$\theta_{13min}$ - $\theta_{13max}]$;$\Delta\theta _{1i}$}    
                        { 
                        \tcc{Cinematica Directa}
                	     $(P_{x},P_{y},P_{z})=forward(\theta _{12},\theta _{13},\theta _{11})$\;
                        \tcc{Jacobiano}
                        $[J_{\theta},J_{x},\theta_{31},\theta_{32},\theta_{33},\theta_{21},\theta_{22},\theta_{23},J] = jacobian.total(P_{x},P_{y},P_{z},\theta_{11},\theta_{12},\theta_{13})   $\;
                        \tcc{Determinantes Jacobiano}
                        $\left|J_{\theta}\right|=det(J_{\theta})$\;
                        $\left|J_{x}\right|=det(J_{x})$\;
                         \tcc{Restricciones}
                          \If {$Restricciones$ $\theta_{2i}$} 
                              {
                                \If {$Restricciones$ $\theta_{3i}$} 
                                      {
                                      \If {$Restricciones$ \left|J_{x}\right| o $\left|J_{\theta}\right|$} 
                                              {
                                              $\mathcal{W}_{(J_{x},J_{\theta})} =(P_{x},P_{y},P_{z},\theta_{11},\theta_{12},\theta_{13})$ \\
                                                  \If {$Restricciones$ $Limites XYZ$} 
                                                  {
                                                  $\mathcal{W}_{(J_{x},J_{\theta},limites)}=(P_{x},P_{y},P_{z},\theta_{11},\theta_{12},\theta_{13})$
                                                  }
                                              }
                                        \Else{$(P_{x},P_{y},P_{z})$ No pertenece al espacio de trabajo\\}
                                      \If {$Restricciones$ \left|J_{x}\right|} 
                                              {
                                               $\mathcal{W}_{J_{x}}=(P_{x},P_{y},P_{z},\theta_{11},\theta_{12},\theta_{13})$
                                              }
                                      \If {$Restricciones$ $\left|J_{\theta}\right|$} 
                                              {
                                               $\mathcal{W}_{J_{\theta}} =(P_{x},P_{y},P_{z},\theta_{11},\theta_{12},\theta_{13})$
                                              }
                                      
                                    }
                              }
                        }  
                    }           
                }
                       \KwRet  $[\mathcal{W}_{(J_{x},J_{\theta})},\mathcal{W}_{J_{x}},\mathcal{W}_{J_{\theta}},\mathcal{W}_{(J_{x},J_{\theta},Limites)}]$\;
            }
\end{algorithm}
    
    \newpage

    \subsection{Trayectoria}
    \newpage



\section{Interfaz de visualización Rviz}
\section{Adams}
\section{Validación de Cinemática de Posición}
\section{Validación de Cinemática de Velocidad}
\section{Validación de Dinámica Inversa}