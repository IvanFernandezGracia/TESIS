\chapter{Introducción}\label{CAP1}

\section{Motivación}

Desde el año 2010 las ventas de robots industriales se han acelerado considerablemente en nuestro país debido a la tendencia continua hacia la automatización y las sucesivas mejoras técnicas en este tipo de sistemas. El estudio de la IDC Worldwide Semiannual Robotics Spending Guide indica que el mercado de la robótica en Latinoamérica presenta un crecimiento en 2019 del 73\% en robots industriales, del 27\% en robots de servicios y de un 0,09\% en robots de consumo. Los robots para producción de alimentos ganan terreno debido a la alta demanda de mano de obra para el trabajo dentro de esta actividad. Sin embargo, este nuevo interés por parte de la industria no se refleja en los estudios y objetivos de las formaciones académicas en las universidades, demostrando nuevamente una de las principales debilidades de nuestro país, la desconexión entre la industria y la academia. Esta desconexión se hace más latente aún con miras a nuestro departamento de mecánica en la USACH, una búsqueda sencilla de la palabra clave “robot” en los repositorios digitales de nuestra universidad, no arroja más de una treintena de resultados posibles, la mayoría de ellos ligados a otros departamentos de la universidad, principalmente eléctrica.

A pesar de esto y con la formación de ingenieros a nuestras espaldas, entendemos que una idea o solución a una problemática, es necesaria adaptarla a la realidad del entorno, en nuestro caso una robótica adaptada a las necesidades y capacidades del país. En este aspecto
Es por esto que se elige entre las múltiples ramas de la robótica, los robots de tipo industrial adoptando la definición de la ISO como “Manipulador multifuncional reprogramable con varios grados de libertad, capaz de manipular materias, piezas, herramientas o dispositivos especiales según trayectorias variables programadas para realizar tareas diversas.” Nos enfocaremos en un robot manipulador de tipo delta, con la esperanza de que en un futuro próximo pueda implementarse y ser de ayuda en la industria agropecuaria de nuestro país, específicamente en sus procesos de packing y pick and place.

Es la intención de esta tesis revertir este escenario en nuestro departamento, y renovar las iniciativas en el ámbito de la robótica en nuestros compañeros presentando una guía completa, de todos los ámbitos necesarios para comenzar a desarrollar un proyecto de robot delta, mostrando y resolviendo las problemáticas asociadas a la cinemática y dinámica del robot, la matemática implícita en estas soluciones y el código mediante el cual se llegó a dichas soluciones. Además, se realizarán simulaciones que comprueben los resultados teóricos. Todo esto bajo un ambiente de software libre y presentando muchos ramales para posibles proyectos en el área.

\section{Hipótesis de estudio}
Es posible realizar el modelamiento de la cinemática y dinámica de un robot delta a través de softwares libres y validarlo por medio de softwares educativos.

\section{Presentación del problema}

\section{Objetivos}

    \subsection{Objetivo general}
        Crear el algoritmo que controla el movimiento de un robot delta y validarlo a través de un software de simulación.
\subsection{Objetivos específicos}
\begin{itemize}
    \item Identificar las piezas de un robot delta y su dimensionamiento en base al espacio de trabajo
    requerido para su aplicación.
    \item Crear un algoritmo que resuelva la cinemática y dinámica tanto directa como inversa del robot delta.
    \item Crear un algoritmo que determine trayectorias cartesianas y angulares a partir del punto inicial y final del efector.
    \item Simular el movimiento del robot delta a través de una herramienta de visualización.
    \item Simular la dinámica del robot delta a través de un software de análisis mecánico.
\end{itemize}

\section{Alcance de la propuesta y limitaciones}
El alcance de este proyecto por tanto incluye diseñar un robot delta, diseñar un algoritmo para el control de su movimiento y verificar los resultados obtenidos a través de software de visualización y modelación.\\ 
Es importante señalar que dentro de los alcances de esta tesis no se incluye la fabricación del robot, optimización de dimensionamiento de piezas y  optimización de trayectoria. Sus resultados serán verificados mediante simulaciones en software.


\section{Estructura de la tesis}

Este documento se organiza en 8 capítulos que, en rasgos generales, se describen a continuación:

\begin{itemize}
    \item {En el Capítulo \eqref{CAP1} se justifica lo importante que es la realización de este tema de tesis para la universidad de Santiago, Chile y el Mundo. Empieza con la motivación principal de la tesis, señalando argumentos relacionados con el ámbito profesional como Ingenieros Mecánicos y personales de los escritores de esta. Posteriormente se presentan antecedentes generales sobre la robótica que validan la realización de esta tesis a través de datos estadísticos y citas de profesionales relacionados con la Universidad de Santiago. Luego se establecen los alcances y limitaciones tanto teóricas como económicas relacionadas con la modelación de un robot. Finalmente presenta el problema a investigar con sus respectivos objetivos generales, objetivos específicos e hipótesis.}
    \item {En el Capítulo \eqref{CAP2} , inicialmente se expone una breve introducción a la robótica explicando su origen y su significado. Posteriormente se presentan los hitos más importantes sobre la robótica en orden cronológico que impactaron en nuestra sociedad. Por otra parte, existen distintas clasificaciones de robot, por lo que se plantean solo las que son más recurrentes en el campo profesional a nivel mundial. Finalmente se destacan datos estadísticos y aplicaciones sobre la robótica en los últimos años.}
    \item {En el Capítulo \eqref{CAP3} se explica detalladamente la solución adoptada que se utiliza para la problemática de la modelación cinemática y dinámica del robot delta. Los campos que se indican en este capítulo son principalmente las piezas mecánicas que se tienen a consideración, software con el que se desarrollan los algoritmos de cinemática y dinámica, interfaz de visualización de las piezas para la validación cinemática y software de simulación para la validación dinámica.}
    \item {El Capítulo \eqref{CAP4} inicia con la presentación de la modelación cinemática de un robot delta, que relaciona las posiciones del espacio articular o actuadores frente a la posición en el espacio cartesiano del efector final. Se explican matemáticamente los siguientes tópicos: cinemática inversa, cinemática directa, Jacobiano, singularidades. Para verificar los algoritmos del jacobiano, se utilizan 2 métodos para su cálculo: vectorial y analítico. }
    \item {En el Capítulo \eqref{CAP5} Se establecen las dimensiones y masas de las partes mecánicas del robot delta con sus respectivos argumentos. La decisión de las dimensiones está estrechamente relacionada con el espacio de trabajo que se necesita para aplicaciones en el sector agrícola (packing de frutas). Se describen los tipos de juntas y sus restricciones.}
    \item {En el Capítulo \eqref{CAP6} se presenta el diagrama de flujo de trabajo con el que se desarrolla la tesis. La idea de este diagrama de flujo es que se vea claramente los pasos que se siguieron en este documento. Se explica por etapa cada parte del diagrama de flujo. Posteriormente se explican los algoritmos de cinemática y dinámica, los pasos para calcular el espacio de trabajo con sus respectivas restricciones y los algoritmos de las trayectorias implementados con el software Robot Operating System (ROS). Se expone el formato de descripción robótica unificada (URDF) de las partes mecánicas del robot delta y la configuración del paquete de visualización ROS visualisation. (Rviz). Finalmente se presenta la simulación dinámica de robot delta en el software ADAMS Student. Se da a conocer la configuración del software, la creación de piezas mecánicas, el tipo de junta entre las piezas, las simplificaciones físicas del problema a solucionar, la trayectoria a realizar y la configuración de los sensores de torque en los actuadores.}
    \item {En el Capítulo \eqref{CAP7} se dan a conocer los resultados obtenidos de la compilación del código de los algoritmos y se visualizan los resultados de las simulaciones a través de ilustraciones y/o gráficos. Se compara los resultados cinemáticos y dinámicos de los modelos con sus respectivos errores relativos entre ellos.  }
    \item {En el Capítulo \eqref{CAP8} se presenta el código de los algoritmos escritos en sus respectivos lenguajes con comentarios en líneas del código para que sean utilizados a futuro fácilmente. Acompañados del código se adjuntan diagramas de flujo en que compilan los códigos.}
    \item {En el Anexo \eqref{anexoB} se desarrollan detalladamente la modelación física y matemática vista en el capítulo V.} 
    \item {En el Anexo \eqref{anexoC} se presenta el código de los algoritmos escritos en sus respectivos lenguajes con comentarios en líneas del código para que sean utilizados a futuro fácilmente. Acompañados del código se adjuntan diagramas de flujo en que compilan los códigos.} 
    \end{itemize}