\chapter{Introducción PULL dos}\label{CAP1}

\section{Motivación}

Desde el año 2010 las ventas de robots industriales se han acelerado considerablemente en nuestro país debido a la tendencia continua hacia la automatización y las sucesivas mejoras técnicas en este tipo de sistemas. El estudio de la IDC Worldwide Semiannual Robotics Spending Guide indica que el mercado de la robótica en Latinoamérica presenta un crecimiento en 2019 del 73\% en robots industriales, del 27\% en robots de servicios y de un 0,09\% en robots de consumo. Los robots para producción de alimentos ganan terreno debido a la alta demanda de mano de obra para el trabajo dentro de esta actividad. Sin embargo, este nuevo interés por parte de la industria no se refleja en los estudios y objetivos de las formaciones académicas en las universidades, demostrando nuevamente una de las principales debilidades de nuestro país, la desconexión entre la industria y la academia. Esta desconexión se hace más latente aún con miras a nuestro departamento de mecánica en la USACH, una búsqueda sencilla de la palabra clave “robot” en los repositorios digitales de nuestra universidad, no arroja más de una treintena de resultados posibles, la mayoría de ellos ligados a otros departamentos de la universidad, principalmente eléctrica.

A pesar de esto y con la formación de ingenieros a nuestras espaldas, entendemos que una idea o solución a una problemática, es necesaria adaptarla a la realidad del entorno, en nuestro caso una robótica adaptada a las necesidades y capacidades del país. En este aspecto   Es por esto que se elige entre las múltiples ramas de la robótica, los robots de tipo industrial adoptando la definición de la ISO como “Manipulador multifuncional reprogramable con varios grados de libertad, capaz de manipular materias, piezas, herramientas o dispositivos especiales según trayectorias variables programadas para realizar tareas diversas.” Nos enfocaremos en un robot manipulador de tipo delta, con la esperanza de que en un futuro próximo pueda implementarse y ser de ayuda en la industria agropecuaria de nuestro país, específicamente en sus procesos de packaging y pick and place.

Es la intención de esta tesis revertir este escenario en nuestro departamento, y renovar las iniciativas en el ámbito de la robótica en nuestros compañeros presentando una guía completa, de todos los ámbitos necesarios para comenzar a desarrollar un proyecto de robot delta, mostrando y resolviendo las problemáticas asociadas a la cinemática y dinámica del robot, la matemática implícita en estas soluciones y el código mediante el cual se llegó a dichas soluciones. Además, se realizarán simulaciones que comprueben los resultados teóricos. Todo esto bajo un ambiente de software libre y presentando muchos ramales para posibles proyectos en el área.

\section{Hipótesis de estudio}

\section{Presentación del problema}

\section{Objetivos}

    \subsection{Objetivo general}
        Crear el algoritmo que controla el movimiento de un robot delta y validarlo a través de un software de simulación.
\subsection{Objetivos específicos}
\begin{itemize}
    \item Identificar las piezas de un robot delta y su dimensionamiento en base al espacio de trabajo
    requerido para su aplicación.
    \item Crear un algoritmo que resuelva la cinemática y dinámica tanto directa como inversa del robot delta.
    \item Crear un algoritmo que determine trayectorias cartesianas y angulares a partir del punto inicial y final del efector.
    \item Simular el movimiento del robot delta a través de una herramienta de visualización.
    \item Simular la dinámica del robot delta a través de un software de análisis mecánico.
\end{itemize}

\section{Alcance de la propuesta y limitaciones}
El alcance de este proyecto por tanto incluye diseñar un robot delta, diseñar un algoritmo para el control de su movimiento y verificar los resultados obtenidos a través de software de visualización y modelación.\\ 
Es importante señalar que dentro de los alcances de esta tesis no se incluye la fabricación del robot, sus resultados serán verificados mediante simulaciones en software.


\section{Estructura de la tesis}

Este documento se organiza en cinco capítulos que, en rasgos generales, se describen a continuación:

\begin{itemize}
    \item En el Capítulo \ref{CAP1}
    \item En el Capítulo \ref{CAP2}
    \item ...
\end{itemize}