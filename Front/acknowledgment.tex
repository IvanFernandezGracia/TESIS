\thispagestyle{fancy}
\paginiciales{Agradecimientos}
\addcontentsline{toc}{chapter}{Agradecimientos}
\vspace{5mm}
Este trabajo de t´ıtulo realizado en la Universidad de Santiago de Chile es un es- fuerzo en el cual, directa o indirectamente, participaron varias personas corrigiendo, opinando,  apoyando,  dando  a´nimos,  acompan˜´andome  en  los  momentos  de  crisis  o felicitando mis logros. Es por esto que en este apartado, me gustar´ıa dedicarle unas palabras a todas estas personas que hicieron posible este trabajo.

En primer lugar, me gustar´ıa agradecer al Dr. Claudio Urrea por ser mi profesor gu´ıa,  quien  a  pesar  de  mi  inconsistencia,  me  apoy´o  para  llevar  a  t´ermino  este  tra- bajo, haci´endome ver que era capaz de grandes cosas, nunca perdiendo de vista mis objetivos y apoyando cada uno de ellos.

Agradecer  al  profesor  Rub´en  Carvajal,  quien  estuvo  conmigo  desde  el  inicio  de este trabajo apoyando mis estudios de la rob´otica y la programaci´on. Me ofreci´o su ayuda en muchas oportunidades y me dedic´o horas de conversaciones, en las cuales me brind´o consejos y vivencias.

A  mis  compan˜eros  de  Ingenier´ıa  Matem´atica  por  ser  un  apoyo  importante  a  lo largo de los an˜os de estudio, fortaleciendo lazos de amistad en el camino.

A mis amigos por ser un pilar fundamental a lo largo de este proceso, brind´ando- me su apoyo incondicionalmente, ayudando a esparcirme en mis momentos de crisis y felicitando cada uno de mis logros.

A mi pareja B´arbara Canales, quien ha estado conmigo desde el inicio de mi pro- ceso de estudios universitarios, participando activamente en cada una de mis etapas (tanto buenas, como malas), apoyando cada una de mis crisis y siendo una de las personas que m´as me impuls´o para que este trabajo fuera posible.

A mis padres Luis Urra y Estela Espinoza, por ser mi sustento emocional, los eternos  preocupados  de  saber  c´omo  me  iba  en  mi  formaci´on  acad´emica,  de  estar cuando fuera necesario y siempre impulsarme a lograr mis metas.
 

A mi hermano Jonathan Urra por ser mi amigo, por estar conmigo a altas horas de la noche cuando necesitaba despejarme de mis estudios, compartiendo hobbies o un simple cigarro. La persona con quien pod´ıa contar para hablar sobre mis proble- mas, d´andome siempre un enfoque distinto.

A mi hija Amelie por ser el u´ltimo empuj´on que necesitaba para llevar a t´ermino este proceso.

Un especial agradecimiento a Leonardo S´anchez, mi profesor de colegio, quien fue el precursor de mi formaci´on matem´atica y me empuj´o a ser la persona que soy hoy en d´ıa.

Gracias a todas estas personas por haber sido parte, directa o indirectamente, de este trabajo de t´ıtulo.

---
En esta sección debe incluir los agradecimientos que Ud. desee, respetando el formato aquí establecido. (Estilo Normal)
No está permitido incluir ningún tipo de declaración que pudiese ser considerada ofensiva.
Sus agradecimientos no deben sobrepasar esta hoja.

Si por alguna razón Ud. no quiere incluir agradecimientos deje esta página en blanco, borrando el título establecido en la plantilla. Si Ud. no incluye Agradecimientos ni Dedicatoria no incluya esta hoja
