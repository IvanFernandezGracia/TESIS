\thispagestyle{fancy}
\paginiciales{ABSTRACT}
\addcontentsline{toc}{chapter}{ABSTRACT}

\vspace{5mm}

This thesis describes, creates and validates the kinematic and dynamic modeling of a delta-type parallel robot with 3 degrees of freedom.

In recent years, the use of kinematics in conjunction with dynamics for the control of parallel robots has been gaining popularity. The use of parallel robots dedicated to 'pick and place' operations opened a series of new perspectives in the domain of fast and precise handling of light objects. These are used in various areas in industry, such as agronomy, manufacturing, pharmaceutical laboratories, etc. 

The problems that are identified about the delta robot and that are solved in this thesis are mainly: robotic systems manage a great complexity, different for each task to be performed and this results in slowness and high costs in the development and implementation of robotics worldwide; robot control schemes based only on position kinematics are not enough for excellent precision and the difficulty of establishing a simple dynamic model that can be easily calculated in real time. Therefore, this work proposes the use of a free middleware oriented to the reuse of code and control of robots; two methods for kinematic and dynamic modeling; workspace where the robot executes tasks; 3D visualization of mechanical parts; trajectory simulation to check the methods and validation of the models by means of educational simulation software.

The results show that the kinematic and dynamic modeling of the two methods give identical values and is valid according to the simulation software. The negligible differences between the results of the models and the simulation software are due to the fact that the latter only approximates the model, in other words, it is not totally identical to the model.
\vfill
\noindent\textbf{Keywords:} Delta Robot, Robot Operating System, ROS, ADAMS, Parallel Robot, Kinematics, Dynamics, Jacobian, Rviz, Workspace. 