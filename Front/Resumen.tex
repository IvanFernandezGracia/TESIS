\thispagestyle{fancy}
\paginiciales{RESUMEN}
\addcontentsline{toc}{chapter}{RESUMEN}
\vspace{5mm}
En el presente trabajo se realiza un análisis numérico del comportamiento dinámico de partículas esféricas rígidas pertenecientes a un nanofluido, siendo agua el fluido base, en régimen laminar y térmicamente desarrollado en microcanales de sección rectangular. Producto de los campos de velocidades, presiones y temperaturas impuestos, las nanopartículas hacen contacto entre ellas y con las paredes.

El presente trabajo es un estudio de los algoritmos de generaci´on y planificaci´on de  trayectorias  para  robots  m´oviles  de  configuraci´on  diferencial  y  su  comparaci´on bajo ´ındices de desempen˜o.

Para  este  trabajo  de  t´ıtulo,  se  aborda  la  configuraci´on  cinem´atica  de  los  robot m´oviles,  controladores  para  el  movimiento  del  robot,  algoritmos  de  planificaci´on  y generaci´on de trayectorias DFS, Hill Climbing y A*, y la complejidad de estos.

Se comparan los algoritmos mediante simulaciones en MatLab y Simulink, pro- gramando una interfaz gr´afica de usuario (GUI) y comparando la distancia, tiempo y velocidad con que los algoritmos obtienen una trayectoria sobre un terreno con obst´aculos.

A partir de los resultados, se concluye que el algoritmo A* es o´ptimo con respecto a la trayectoria generada, el algoritmo DFS obtiene las trayectorias en menor tiempo y el algoritmo Hill Climbing genera trayectorias de menor distancia que DFS y en menor tiempo que A*.

Presentación v
PRESENTACIÓN
El presente trabajo se centra en el estudio de los componentes léxico y
morfosintáctico del lenguaje en pacientes afásicos bilingües del castellano y del
catalán. En este sentido, es relevante destacar dos cuestiones. Por un lado, la
importancia de la investigación relacionada con dichos componentes en el estudio
del procesamiento del lenguaje en pacientes agramáticos. A lo largo de la
literatura se puede encontrar una gran variedad de trabajos que han aportado
información acerca de las alteraciones que presentan los pacientes afásicos
agramáticos en lenguas como el inglés o el italiano. Sin embargo, los déficits que
muestran los pacientes agramáticos en lenguas como el castellano o el catalán han
estado mucho menos estudiados. Por ello, el presente trabajo pretende contribuir
en el conocimiento de cuáles son las dificultades que presentan estos pacientes
agramáticos y si éstas se pueden considerar similares a las presentadas por
pacientes de otras lenguas.
Por otro lado, el interés por el estudio del fenómeno del bilingüismo. No
existen muchos trabajos que relacionen y comparen de una manera detallada y
precisa las alteraciones de los afásicos agramáticos en castellano y en catalán. Por
ejemplo, uno de los objetivos específicos de este trabajo ha sido comprobar si los
pacientes que aquí se han estudiado presentaban patrones de recuperación
similares en castellano y en catalán, por tratarse de lenguas con una estructura
similar. O si, por el contrario, las formas de recuperación presentadas por los
pacientes en ambas lenguas difieren de una forma clara. Ambas cuestiones serán
examinadas tanto en el ámbito de la producción como en el de la comprensión,
aunque haciendo mayor énfasis en el primero.
Para llevar a cabo este estudio, adoptaremos el enfoque de la
Neuropsicología cognitiva, utilizando la metodología de estudio de casos, con la
finalidad principal de estudiar los distintos tipos de alteraciones que presentan los
pacientes. Asimismo, con esta metodología se pretende encontrar disociaciones
(simples o dobles) que contribuyan a estructurar un modelo de procesamiento del
lenguaje en personas sin daño cerebral.
Con el fin de conseguir este objetivo, el presente trabajo se ha estructurado
en seis capítulos además de los apéndices y la bibliografía.
vi Estudio del componente léxico y morfosintáctico en pacientes afásicos bilingües
En el capítulo I, se presenta, a modo de introducción general, un breve
recorrido histórico sobre el tema de la afasia. Dicho recorrido tiene como punto de
inicio los primeros documentos médicos escritos encontrados en los que se
recogían manifestaciones clínicas lingüísticas, pasando por los inicios de la
Neuropsicología Clásica y el localizacionismo con autores destacados como Paul
Broca y Karl Wercnicke; y hasta llegar al desarrollo de la Neuropsicología
Cognitiva en nuestros días.
El capítulo II se compone de tres apartados principales. En el primero, se
aborda el tema del agramatismo desde el enfoque de la Neuropsicología
Cognitiva, presentando una revisión de los trabajos que han investigado acerca de
la sintomatología de los pacientes agramáticos tanto en el ámbito de la producción
como en el de la comprensión. Asimismo, se hace mención de la polémica que
existe actualmente en relación al concepto de agramatismo debido a la
variabilidad de síntomas que presentan los pacientes agramáticos. En el segundo
apartado, debido a que nuestros pacientes presentaron anomia como síntoma en
mayor o menor grado, se incluye una revisión de algunos de los trabajos que han
estudiado la sintomatología de los pacientes anómicos. Y, en tercer y último lugar,
se presenta un apartado dedicado a los modelos teóricos de procesamiento del
lenguaje propuestos por diferentes autores y bajo los cuales se ha pretendido dar
una explicación a la sintomatología afásica.
En el capítulo III se especifica la relación existente entre afasia y
bilingüismo a partir de trabajos de autores como Fabbro o Paradis que han
estudiado a pacientes afásicos bilingües (o multilingües) de distintos dialectos del
italiano o incluso del francés y del inglés. Además, se abordan cuestiones como
por ejemplo los patrones de recuperación de los pacientes o la representación
cerebral de las distintas lenguas.
En el capítulo IV se presentarán los aspectos relacionados con los
objetivos del presente trabajo, así como la metodología empleada. Dentro de la
metodología, se especifican los pacientes estudiados y las tareas que han sido
diseñadas para su administración. Se incluye también el tipo de análisis que
hemos elaborado para clasificar los errores de nuestros pacientes en las diversas
tareas. Si bien hubo un grupo determinado de tareas que se administraron a todos
los pacientes de forma general, otras tareas no pudieron pasarse a todos los
Presentación vii
pacientes debido a las dificultades que tuvimos para acceder a algunos de ellos, ya
fuera por recaídas de la enfermedad como por la negación del paciente a seguir
colaborando.
El capítulo V corresponde a los resultados obtenidos en el análisis de las
tareas realizadas por los pacientes que han participado en el presente trabajo.
Puesto que tres de los cinco pacientes que formaron parte de la muestra eran
bilingües, la descripción de los resultados de estos pacientes se realizó en dos
bloques. En un primer bloque se describen los resultados obtenidos a partir de la
administración de las tareas en la primera lengua del paciente (que en todos los
casos coincidió ser el catalán), y en un segundo bloque se presentan los resultados
obtenidos a partir del análisis de las tareas en la segunda lengua de los pacientes
(castellano). En la exposición de cada paciente se ha incluido una breve
presentación de la historia médica y social, los resultados obtenidos en una
primera evaluación clínica del lenguaje y, finalmente el análisis de las pruebas
experimentales. Si bien en este trabajo se ha prestado una atención especial al
análisis cualitativo de los datos por tratarse de una metodología de estudio de caso
único, también se han realizado análisis cuantitativos. Asimismo, en este capítulo
se incluye la discusión de los resultados obtenidos para cada paciente en base a
los modelos teóricos anteriormente mencionados.
En el capítulo VI y último, se presentan las principales conclusiones
generales que se han obtenido a partir de la realización de este trabajo.
\vfill
\noindent\textbf{Palabras clave:} Robot Delta, Robot Operating System, ROS, ADAMS, Robot Paralelo, Cinematica, Dinamica, Jacobiano, Rviz, Espacio de Trabajo, Robótica.
