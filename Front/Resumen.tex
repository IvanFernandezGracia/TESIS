\thispagestyle{fancy}
\paginiciales{RESUMEN}
\addcontentsline{toc}{chapter}{RESUMEN}
\vspace{5mm}
En el presente trabajo se realiza un análisis numérico del comportamiento dinámico de partículas esféricas rígidas pertenecientes a un nanofluido, siendo agua el fluido base, en régimen laminar y térmicamente desarrollado en microcanales de sección rectangular. Producto de los campos de velocidades, presiones y temperaturas impuestos, las nanopartículas hacen contacto entre ellas y con las paredes.

El presente trabajo es un estudio de los algoritmos de generaci´on y planificaci´on de  trayectorias  para  robots  m´oviles  de  configuraci´on  diferencial  y  su  comparaci´on bajo ´ındices de desempen˜o.

Para  este  trabajo  de  t´ıtulo,  se  aborda  la  configuraci´on  cinem´atica  de  los  robot m´oviles,  controladores  para  el  movimiento  del  robot,  algoritmos  de  planificaci´on  y generaci´on de trayectorias DFS, Hill Climbing y A*, y la complejidad de estos.

Se comparan los algoritmos mediante simulaciones en MatLab y Simulink, pro- gramando una interfaz gr´afica de usuario (GUI) y comparando la distancia, tiempo y velocidad con que los algoritmos obtienen una trayectoria sobre un terreno con obst´aculos.

A partir de los resultados, se concluye que el algoritmo A* es o´ptimo con respecto a la trayectoria generada, el algoritmo DFS obtiene las trayectorias en menor tiempo y el algoritmo Hill Climbing genera trayectorias de menor distancia que DFS y en menor tiempo que A*.


\vfill
\noindent\textbf{Palabras clave:} Robot Delta, Robot Operating System, ROS, ADAMS, Robot Paralelo, Cinematica, Dinamica, Jacobiano, Rviz, Espacio de Trabajo, Robótica.
